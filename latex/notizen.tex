\documentclass{article}

%%% Defines %%%%%%%%%%%%%%%%%%%%%%%%%%%%%%%%%%%%%%%%%%%%%%%%%%%%%%%%%%%%%%%%%%%%

\usepackage{graphicx}
\usepackage{tikz}
\usepackage{pgfplots}
\pgfplotsset{compat=1.18}
\usetikzlibrary{shapes.misc}
\usepackage[european,straightvoltages]{circuitikz}
\usepackage[ngerman]{babel} 
\usepackage[colorlinks]{hyperref}
\usepackage{caption, float, subcaption}
\usepackage{xcolor}
\usepackage{setspace}
\usepackage{mathtools, amssymb, ntheorem, amsmath, siunitx}
\sisetup{per-mode=fraction, separate-uncertainty=true,exponent-base=10,output-decimal-marker={,}}
\DeclareSIUnit\liter{l}
\usepackage{enumitem}
\usepackage{minted}

\usepackage{tikz-uml}

% maths
\newcommand{\abs}[1]{\left| #1 \right|}
\newcommand{\br}[1]{\left( #1 \right)}
\newcommand{\ubar}[1]{\mkern 1.5mu\underline{\mkern-1.5mu#1\mkern-1.5mu}\mkern 1.5mu}

% image
\newcommand{\img}[5]{
    \begin{figure} [#5]
    \centering
    \includegraphics[width=#2\linewidth]{#1}
    \caption{#3}
    \label{pic:#4}
    \end{figure}
}

% tikz
\newenvironment{gfx}[3]
{
    \newcommand{\gfxname}{#2}
    \newcommand{\gfxcaption}{#1}
    \begin{figure} [#3]
    \centering
    \begin{tikzpicture}
}
{
    \end{tikzpicture}
    \caption{\gfxcaption}
    \label{gfx:\gfxname}
    \end{figure}
}

\newcommand{\opt}{\ensuremath{\parallel}}

% cicuitikz
\newenvironment{ckt}[3]
{
    \newcommand{\cktname}{#2}
    \newcommand{\cktcaption}{#1}
    \begin{figure} [#3]
    \centering
    \begin{circuitikz}
	\draw
}
{
	;
    \end{circuitikz}
    \caption{\cktcaption}
    \label{ckt:\cktname}
    \end{figure}
}

% Code
\newfloat{Code}{htbp}{loc}
\floatname{Code}{Quelltext}
\definecolor{LightGray}{gray}{0.9}
\newenvironment{code}[4]
{%
  \VerbatimEnvironment
  \begin{Code} [#4]
  \caption{#2}%
  \label{cod:#3}%
  \begin{minted}[frame=lines,framesep=2mm,baselinestretch=1.2,bgcolor=LightGray,fontsize=\footnotesize,style=emacs]{#1}%
}
{%
  \end{minted}%
  \vspace{-20pt}%
  \end{Code}%
}
\makeatletter
\AtBeginEnvironment{minted}{\dontdofcolorbox}
\def\dontdofcolorbox{\renewcommand\fcolorbox[4][]{##4}}
\makeatother
\newcommand{\listofcode}{
  \doublespacing
  \listof{Code}{Quelltextverzeichnis}
}

%%% Head %%%%%%%%%%%%%%%%%%%%%%%%%%%%%%%%%%%%%%%%%%%%%%%%%%%%%%%%%%%%%%%%%%%%%%%

\title{Title}
\author{Benjamin Brohs, Kevin Keheck, Justin Meng}
\date{Date}

\setlength{\parindent}{0pt}
\setlength{\parskip}{1em}

\begin{document}

%%% Title %%%%%%%%%%%%%%%%%%%%%%%%%%%%%%%%%%%%%%%%%%%%%%%%%%%%%%%%%%%%%%%%%%%%%%

\begin{titlepage}
  \centering
	\begin{tabular}{lcr}
		\includegraphics[width=0.35\textwidth]{fachbereich.png} & \hspace{0.195\textwidth} & \includegraphics[width=0.35\textwidth]{Q04_HTW_Berlin_Logo_quer_pos_FARBIG_RGB.jpg}\\
	\end{tabular}	
	\\[3cm]
	\Large
	Belegarbeit\\
	\vspace{2cm}
	\textbf{Minecraft-One-Week-Challenge Reloaded}\\
	\vspace{2cm}
	\begin{tabular}{ll} 
		Im Studiengang: & Computer Engineering \\		
	\end{tabular}	
	\\[3cm]
	\normalsize
	\begin{tabular}{ll}
	      \textbf{Erstellt von:} & Benjamin Brohs, Kevin Keheck, Justin Meng \\
        \textbf{Modul:} & Softwaretechnik \\
        \textbf{Semester:} & Sommersemester 2025 \\
	\textbf{Dozent:} & Thomas Baar
	\end{tabular}	
\end{titlepage}

\tableofcontents

\newpage

%%% Inhalt %%%%%%%%%%%%%%%%%%%%%%%%%%%%%%%%%%%%%%%%%%%%%%%%%%%%%%%%%%%%%%%%%%%

\section{Meilenstein 1} \label{sec:ms1}

Meilenstein 1 befasst sich mit der Beschreibung der Software, wie sie ohne die Änderung aufgebaut ist. Sowohl textuell, als auch per Modell.

\subsection{Informelle Beschreibung} \label{subsec:inf}

Grundlegend basiert diese Softftware auf dem Spiel \href{https://www.minecraft.net}{Minecraft}. Bei der vorgestellen Software handelt es sich um eine wesentlich ausgedünnte Version dessen. Der Sinn des Spiels ist, dass ein Spieler sich in einer Welt, generiert aus Blöcken verschiedener Texturen, überleben soll und kreativ ausleben kann. Der Spieler hat dabei die Möglichkeit zu Springen, zu Sprinten und zu Schleichen um diese Welt zu erkunden. Außerdem kann der Spiler diese Welt manipulieren, indem er Blöcke abbaut und sie woanders wieder platzieren kann. Die generierte Welt ist groß genug, sodass sich der Spieler in ihr frei bewegen kann. Dabei stößt der Spieler auf verschiedenste generierte Strukturen und Biome. Das Original Open-Source-Projekt ist zu finden unter \url{https://github.com/Hopson97/MineCraft-One-Week-Challenge}.

\subsection{Use-Case-Modell} \label{subsec:usecase}

\begin{center}
\begin{tikzpicture}
  % Systemgrenze
  \begin{umlsystem}[x=4, y=0]{Minecraft-Klon}
    \umlusecase[x=0, y=0]{Bewegen}
    \umlusecase[x=0, y=-1.5]{Springen}
    \umlusecase[x=0, y=-3]{Flugmodus umschalten}
    \umlusecase[x=0, y=-4.5]{Maussteuerung umschalten}
    \umlusecase[x=0, y=-6]{Block abbauen}
    \umlusecase[x=0, y=-7.5]{Block platzieren}
  \end{umlsystem}

  % Akteur
  \umlactor[x=-2, y=-4.5]{Spieler}

  % Assoziationen
  \umlassoc{Spieler}{usecase-1}
  \umlassoc{Spieler}{usecase-2}
  \umlassoc{Spieler}{usecase-3}
  \umlassoc{Spieler}{usecase-4}
  \umlassoc{Spieler}{usecase-5}
  \umlassoc{Spieler}{usecase-6}
\end{tikzpicture}
\end{center}

\subsubsection*{Physikalische Rahmenbedingungen der Spielfigur}

Die Bewegung der Spielfigur basiert auf einem kontinuierlichen Physikmodell, das zwischen horizontaler und vertikaler Bewegung sowie zwischen Flug- und Bodenmodus unterscheidet. Die Geschwindigkeit in jeder Raumrichtung wird in jedem Simulationsschritt auf Basis folgender Prinzipien berechnet:

\begin{itemize}
  \item \textbf{Horizontale Bewegung:}  Eingaben wie Bewegungstasten erzeugen eine gerichtete beschleunigung basierend auf der Blickrichtung der Spielfigur. Diese wird auf die aktuelle Geschwindigkeit addiert. Unabhängig vom Zustand (am Boden, in der Luft, im Flugmodus) wird eine konstante Dämpfung angewendet, die die Geschwindigkeit exponentiell reduziert.

  \item \textbf{Vertikale Bewegung:}  Die vertikale Bewegung unterscheidet sich je nach Modus, im \textbf{Flugmodus} gilt die gleiche Dämpfung wie in horizontaler Richtung. Die Eingaben \texttt{Springen} (Leertaste) und \texttt{Schleichen} (Strg) erzeugen eine gerichtete Beschleunigung nach oben oder unten, analog zu Bewegungseingaben horizontal. Im \textbf{Bodenmodus} gilt diese Dämpfung nicht mehr, stattdessen wird eine konstante Gravitationskonstante von der vertikalen Geschwindigkeit subtrahiert. Ein Sprung setzt dann die vertikale Geschwindigkeit ohne eine Beschleunigung auf einen festen positiven Wert.

  \item \textbf{Zusammenfassung der Logik:}
  \begin{itemize}
    \item Dämpfung wirkt \textbf{immer} in allen Richtungen außer vertikal im Bodenmodus.
    \item Gravitation ersetzt im Bodenmodus die vertikale Dämpfung.
    \item Sprung ist eine \textit{diskrete Initialisierung der vertikalen Geschwindigkeit}, kein kontinuierlicher Prozess.
    \item Die Dämpfung erfolgt exponentiell, die Gravitation linear.
  \end{itemize}
\end{itemize}

\textit{Hinweis:} Diese Mechaniken gelten systemweit und werden innerhalb der Use Cases nicht vollständig modelliert, sondern vorausgesetzt. Ihre konkrete Auswirkung auf das Spielerlebnis ergibt sich aus dem Zusammenspiel dieser physikalischen Regeln mit den Eingaben des Spielers.



\subsubsection*{UC01 – Bewegen}

\textbf{Name:} Bewegen \\
\textbf{Akteur:} Spieler \\
\textbf{Ziel:} Fortbewegung der Spielfigur in der Spielwelt entsprechend der Eingaben \\
\textbf{Vorbedingungen:} Spiel läuft, Spielfigur ist aktiv \\
\textbf{Nachbedingungen:} Position der Spielfigur wurde ggf. angepasst \\
\textbf{Beschreibung:} Der Spieler bewegt die Spielfigur durch Eingabe von Bewegungstasten. Die Bewegungsrichtung wird durch die aktuelle Blickrichtung bestimmt. Der Use Case bleibt aktiv, solange Eingaben erfolgen. Sprinten, Schleichen sowie vertikale Flugbewegung (Flugmodus aktiv) beeinflussen die Fortbewegung.

\textbf{Ablaufspezifikation:}
\begin{description}[style=nextline,leftmargin=1.9cm,labelwidth=1.6cm]
  \item[1.] Spieler passt die Blickrichtung an (Mausbewegung)
  \item[1a.] Spieler behält Blickrichtung bei
  \item[2.] Spieler drückt Bewegungstaste (W/A/S/D)
  \item[2a.] Speiler tätigt keine Eingabe - Use Case bleibt aktiv
  \item[2a.1.] Wiederhole Schritt 1
  \item[2\opt b.] Spieler drückt Sprinttaste (Shift)
  \item[2\opt c.] Spieler drückt Schleichtaste (Strg)
  \item[3.] System berechnet neue Position der Spielfigur
  \item[3\opt a.] Spieler ist am Sprinten
  \item[3\opt a.1.] Geschwindigkeit der Spielfigur wird erhöht
  \item[3\opt b.] Spieler ist am Schleichen
  \item[3\opt b.1.] Geschwindigkeit der Spielfigur wird verringert
  \item[3\opt b.1a.] Spieler ist am fliegen.
  \item[3\opt b.1a.1.] Spielfigur bekommt Geschwindigkeit nach unten
  \item[4.] Spiel aktualisiert Position der Spielfigur
  \item[5.] Wiederhole Schritt 1
\end{description}

\subsubsection*{UC02 – Springen}

\textbf{Name:} Springen \\
\textbf{Akteur:} Spieler \\
\textbf{Ziel:} Hindernisse überwinden oder Höhe gewinnen \\
\textbf{Vorbedingungen:} Spielfigur befindet sich in der Spielwelt \\
\textbf{Nachbedingungen:} Spielfigur bewegt sich vertikal nach oben oder bleibt unverändert \\
\textbf{Beschreibung:} Durch Drücken der Springtaste kann die Spielfigur entweder einen einmaligen Sprung ausführen (bei Bodenkontakt) oder im Flugmodus kontinuierlich aufsteigen. Das Verhalten hängt vom Zustand der Spielfigur (am Boden oder im Flug) ab.

\textit{Hinweis: Im Flugmodus verhält sich der Use Case wie eine kontinuierliche Bewegungseingabe. Außerhalb des Flugmodus handelt es sich um eine einmalige Aktion.}

\textbf{Ablaufspezifikation:}
\begin{description}[style=nextline,leftmargin=1.9cm,labelwidth=1.6cm]
  \item[1.] Spieler drückt Springtaste (Leertaste)
  \item[2.] System validiert, dass Spieler springen kann
  \item[2a.] Spielfigur hat keinen Bodenkontakt
  \item[2a.1.] UseCase endet erfolglos
  \item[2b.] Spielfigur ist im Flugmodus
  \item[2b.1.] Solange Springtaste gedrückt ist, erhält die Spielfigur Bewegung nach oben
  \item[2b.2.] Der Use Case endet erfolgreich, sobald die Taste losgelassen wird
  \item[3.] Sprungbewegung wird ausgelöst
  \item[4.] UseCase endet erfolgreich
\end{description}

\subsubsection*{UC03 – Flugmodus umschalten}

\textbf{Name:} Flugmodus umschalten \\
\textbf{Akteur:} Spieler \\
\textbf{Ziel:} Zustand der Spielfigur zwischen Boden- und Flugmodus wechseln \\
\textbf{Vorbedingungen:} Spielfigur befindet sich in der Spielwelt \\
\textbf{Nachbedingungen:} Gravitation ist deaktiviert oder wiederhergestellt; Flugverhalten wird angepasst \\
\textbf{Beschreibung:} Durch Drücken der Flugmodus-Taste kann der Spieler zwischen normalem Bewegungsmodus und Flugmodus wechseln. Der Flugmodus deaktiviert die Gravitation und ermöglicht kontrollierte Bewegung in alle Richtungen. Beim erneuten Drücken wird der Zustand wieder zurückgesetzt.

\textbf{Ablaufspezifikation:}
\begin{description}[style=nextline,leftmargin=1.9cm,labelwidth=1.6cm]
  \item[1.] Spieler drückt die Fliegentaste (F)
  \item[2.] Flugmodus wird aktiviert
  \item[2a.] Flugmodus ist bereits aktiv
  \item[2a.1.] Flugmodus wird deaktiviert
  \item[3.] UseCase endet erfolgreich
\end{description}

\subsubsection*{UC04 – Maussteuerung umschalten}

\textbf{Name:} Maussteuerung umschalten \\
\textbf{Akteur:} Spieler \\
\textbf{Ziel:} Spielsteuerung temporär deaktivieren (z.B. für Fensterwechsel) \\
\textbf{Vorbedingungen:} Spielsteuerung ist aktiv \\
\textbf{Nachbedingungen:} Mauszeiger ist sichtbar, Steuerung pausiert \\
\textbf{Beschreibung:} Mit einer Taste kann der Spieler zwischen Maussteuerung und normaler Cursorbewegung wechseln.

\textbf{Ablaufspezifikation:}
\begin{description}[style=nextline,leftmargin=1.9cm,labelwidth=1.6cm]
  \item[1.] Spieler drückt Mausfreigabetaste (L)
  \item[2.] System schaltet Mauszeiger frei
  \item[2a.] Spieler kann Mauszeiger bewegen
  \item[2a.1.] Mauszeiger wird eingefangen
  \item[2a.2.] Kamera- und Bewegungssteuerung werden aktiviert
  \item[2a.3.] UseCase endet erfolglos
  \item[3.] Kamera- und Bewegungssteuerung werden deaktiviert
  \item[3\opt a.] Spieler kann Fenster verlassen
  \item[4.] UseCase endet erfolgreich
\end{description}

\subsubsection*{UC05 – Block abbauen}

\textbf{Name:} Block abbauen \\
\textbf{Akteur:} Spieler \\
\textbf{Ziel:} Entfernen eines Blocks aus der Spielwelt \\
\textbf{Vorbedingungen:} Spieler sieht einen erreichbaren Block \\
\textbf{Nachbedingungen:} Block ist nicht mehr in der Welt vorhanden \\
\textbf{Beschreibung:} Durch Mausklick kann ein Block entfernt und ggf. aufgenommen werden.

\textbf{Ablaufspezifikation:}
\begin{description}[style=nextline,leftmargin=1.9cm,labelwidth=1.6cm]
  \item[1.] Spieler passt die Blickrichtung an (Mausbewegung)
  \item[2.] Spieler klickt die linken Maustaste
  \item[3.] System validiert, dass der Block abgebaut werden kann
  \item[3a.] Es ist kein entfernbarer Block im Sichtfeld
  \item[3a.1.] UseCase endet erfolglos
  \item[4.] Block wird aus der Welt entfernt
  \item[3\opt a.] Partikeleffekt beim Abbauen anzeigen
  \item[5.] Block wird dem Inventar hinzugefügt
  \item[5a.] Inventar ist voll
  \item[5a.1.] UseCase endet erfolglos 
  \item[6.] UseCase endet erfolgreich
\end{description}

\subsubsection*{UC06 – Block platzieren}

\textbf{Name:} Block platzieren \\
\textbf{Akteur:} Spieler \\
\textbf{Ziel:} Platzieren eines Blocks an einer bestimmten Stelle \\
\textbf{Vorbedingungen:} Ein Block ist ausgewählt \\
\textbf{Nachbedingungen:} Der Block ist in der Spielwelt sichtbar \\
\textbf{Beschreibung:} Spieler kann Blöcke frei setzen, um die Spielwelt zu gestalten.

\textbf{Ablaufspezifikation:}
\begin{description}[style=nextline,leftmargin=1.9cm,labelwidth=1.6cm]
  \item[1.] Spieler passt die Blickrichtung an (Mausbewegung)
  \item[2.] Spieler klickt die linken Maustaste
  \item[3.] System validiert, dass der Block platziert werden kann
  \item[3a.] Es ist keine valide Oberfläche im Sichtfeld
  \item[3a.1.] UseCase endet erfolglos
  \item[4.] Block wird an der Stelle platziert
  \item[4a.] Spieler hat keinen Block im Inventar
  \item[4a.1.] UseCase endet erfolglos
  \item[5.] Block wird aus dem Inventar entfernt
  \item[6.] UseCase endet erfolgreich
\end{description}







\subsection{Domänenmodell} \label{subsec:domain}

%%% Verzeichnisse %%%%%%%%%%%%%%%%%%%%%%%%%%%%%%%%%%%%%%%%%%%%%%%%%%%%%%%%%%%%%%
\section{Anhang}

In diesem Abschnitt finden sich verschiedene Verzeichnisse der genutzten 
Ressourcen, sowie zusätzliche Anhänge von Ressourcen, die nicht direkt im 
Hauptteil des Dokuments enthalten sind.

\listoffigures
\listoftables
\listofcode

\iffalse
\section*{Arbeitsverteilung}

\subsection*{Benjamin Brohs}
\begin{mylist}
  \mylistentry{sec:intro}
\end{mylist}
\fi

%%% Anhänge %%%%%%%%%%%%%%%%%%%%%%%%%%%%%%%%%%%%%%%%%%%%%%%%%%%%%%%%%%%%%%%%%%%%

\iffalse
\begin{table}[h]
	\centering
	\begin{tabular}{|c|c|}
	\hline
	$U_{AB}$ & I \\
	\hline
	0   & 0 \\
	0,1 & ~ \\
	\hline
	\end{tabular}
	\caption{Messwerte nach 3.3.1}
	\label{tab:rx}
\end{table}
\fi

\iffalse
\begin{gfx}{Bodediagramm Phasengang Bandpass}{bodeband2}{}
    \begin{axis}
        [
        xlabel={$f$ [Hz] },
        ylabel={Phasenunterschied $\Delta \varphi_{12}$ [°] },
        grid=both,
        grid style={line width=.1pt, draw=gray!10},
        major grid style={line width=.2pt,draw=gray!50},
        xmin=20, xmax=100000,
        ymin=-90, ymax=90,
        width=11.89cm, height=6cm,
        every axis plot/.append style={very thick, mark=*},
        legend style={at={(0.5,-0.1)}, anchor=north, legend columns=-1},
        xtick={10, 100, 1000, 10000, 100000},
        ytick={-90,-75,-60,-45,-30,-15,0,15,30,45,60,75,90},
        xmode=log
        ]

        \addplot[
        scatter,
        only marks,
        point meta=explicit symbolic
        ] table[meta=label] {
            x y label
            100 74 1
            200 59 2
            341.49 44 10
            500 32 3
            1000 14 4
            2000 -1 5
            5000 -20 6
            10000 -39 7
            12847.92 -46 11
            20000 -59 8
            50000 -76 9
        };

        \addplot[
        no marks,
        thin,
        ] coordinates{
            (20, 86.51900995)
            (23.79683391, 85.86013154)
            (28.3144652, 85.07750427)
            (33.68973128, 84.1485388)
            (40.08544699, 83.04696273)
            (47.6953362, 81.74251804)
            (56.74989969, 80.20084926)
            (67.52339687, 78.38375782)
            (80.34215301, 76.25009007)
            (95.59444355, 73.7576336)
            (113.7422548, 70.86648447)
            (135.3352773, 67.54432971)
            (161.0275558, 63.7738094)
            (191.5972999, 59.56138864)
            (227.9704562, 54.94591252)
            (271.2487541, 50.00361115)
            (322.7430775, 44.84581114)
            (384.0131705, 39.60725845)
            (456.9148818, 34.42684488)
            (543.6563777, 29.42650423)
            (646.8650261, 24.69507184)
            (769.6669794, 20.2811201)
            (915.7818637, 16.1944633)
            (1089.635445, 12.41311546)
            (1296.493686, 8.892029277)
            (1542.622245, 5.571078989)
            (1835.476267, 2.381199607)
            (2183.926194, -0.751349392)
            (2598.526445, -3.902071813)
            (3091.835111, -7.145930748)
            (3678.794331, -10.554851)
            (4377.182883, -14.19424224)
            (5208.154703, -18.11787564)
            (6196.879622, -22.3608024)
            (7373.305756, -26.93078726)
            (8773.066621, -31.80001815)
            (10438.56046, -36.90028499)
            (12420.23448, -42.12536058)
            (14778.11285, -47.34270134)
            (17583.61485, -52.41271888)
            (20921.7181, -57.2098854)
            (24893.53254, -61.63886918)
            (29619.36296, -65.64164712)
            (35242.35304, -69.19586452)
            (41932.82109, -72.30765117)
            (49893.41894, -75.00259162)
            (59365.27018, -77.317439)
            (70635.27373, -79.29373889)
            (84044.79385, -80.97351325)
            (100000, -82.3966604)
        };

        \addplot [densely dotted, thin, no marks] coordinates {(341.485847695166,-90) (341.485847695166,90)};
        \node [below left] at (axis cs:341.485847695166,43.14655286) {$f_h$};
        
        \addplot [densely dotted, thin, no marks] coordinates {(2094.60836705993,-90) (2094.60836705993,90)};
        \node [below left] at (axis cs:2094.60836705993,0) {$f_0$};
        
        \addplot [densely dotted, thin, no marks] coordinates {(12847.9239797778,-90) (12847.9239797778,90)};
        \node [above right] at (axis cs:12847.9239797778,-43.14655286) {$f_t$};
    \end{axis}
\end{gfx}
\fi

%%%%%%%%%%%%%%%%%%%%%%%%%%%%%%%%%%%%%%%%%%%%%%%%%%%%%%%%%%%%%%%%%%%%%%%%%%%%%%%%

\end{document}