\documentclass{article}

%%% Defines %%%%%%%%%%%%%%%%%%%%%%%%%%%%%%%%%%%%%%%%%%%%%%%%%%%%%%%%%%%%%%%%%%%%

\usepackage{graphicx}
\usepackage{tikz}
\usepackage{pgfplots}
\pgfplotsset{compat=1.18}
\usetikzlibrary{shapes.misc}
\usepackage[european,straightvoltages]{circuitikz}
\usepackage[ngerman]{babel} 
\usepackage[colorlinks]{hyperref}
\usepackage{caption, float, subcaption}
\usepackage{xcolor}
\usepackage{setspace}
\usepackage{mathtools, amssymb, ntheorem, amsmath, siunitx}
\sisetup{per-mode=fraction, separate-uncertainty=true,exponent-base=10,output-decimal-marker={,}}
\DeclareSIUnit\liter{l}
\usepackage{enumitem}
\usepackage{minted}

% maths
\newcommand{\abs}[1]{\left| #1 \right|}
\newcommand{\br}[1]{\left( #1 \right)}
\newcommand{\ubar}[1]{\mkern 1.5mu\underline{\mkern-1.5mu#1\mkern-1.5mu}\mkern 1.5mu}

% image
\newcommand{\img}[5]{
    \begin{figure} [#5]
    \centering
    \includegraphics[width=#2\linewidth]{#1}
    \caption{#3}
    \label{pic:#4}
    \end{figure}
}

% tikz
\newenvironment{gfx}[3]
{
    \newcommand{\gfxname}{#2}
    \newcommand{\gfxcaption}{#1}
    \begin{figure} [#3]
    \centering
    \begin{tikzpicture}
}
{
    \end{tikzpicture}
    \caption{\gfxcaption}
    \label{gfx:\gfxname}
    \end{figure}
}


% cicuitikz
\newenvironment{ckt}[3]
{
    \newcommand{\cktname}{#2}
    \newcommand{\cktcaption}{#1}
    \begin{figure} [#3]
    \centering
    \begin{circuitikz}
	\draw
}
{
	;
    \end{circuitikz}
    \caption{\cktcaption}
    \label{ckt:\cktname}
    \end{figure}
}

% Code
\newfloat{Code}{htbp}{loc}
\floatname{Code}{Quelltext}
\definecolor{LightGray}{gray}{0.9}
\newenvironment{code}[4]
{%
  \VerbatimEnvironment
  \begin{Code} [#4]
  \caption{#2}%
  \label{cod:#3}%
  \begin{minted}[frame=lines,framesep=2mm,baselinestretch=1.2,bgcolor=LightGray,fontsize=\footnotesize,style=emacs]{#1}%
}
{%
  \end{minted}%
  \vspace{-20pt}%
  \end{Code}%
}
\makeatletter
\AtBeginEnvironment{minted}{\dontdofcolorbox}
\def\dontdofcolorbox{\renewcommand\fcolorbox[4][]{##4}}
\makeatother
\newcommand{\listofcode}{
  \doublespacing
  \listof{Code}{Quelltextverzeichnis}
}

%%% Head %%%%%%%%%%%%%%%%%%%%%%%%%%%%%%%%%%%%%%%%%%%%%%%%%%%%%%%%%%%%%%%%%%%%%%%

\title{Title}
\author{Benjamin Brohs}
\date{Date}

\setlength{\parindent}{0pt}
\setlength{\parskip}{1em}

\begin{document}

%%% Title %%%%%%%%%%%%%%%%%%%%%%%%%%%%%%%%%%%%%%%%%%%%%%%%%%%%%%%%%%%%%%%%%%%%%%

\begin{titlepage}
  \centering
	\begin{tabular}{lcr}
		\includegraphics[width=0.35\textwidth]{fachbereich.png} & \hspace{0.195\textwidth} & \includegraphics[width=0.35\textwidth]{Q04_HTW_Berlin_Logo_quer_pos_FARBIG_RGB.jpg}\\
	\end{tabular}	
	\\[3cm]
	\Large
	Versuchsprotokoll\\
	\vspace{2cm}
	\textbf{Experiment Dichte}\\
	\vspace{2cm}
	\begin{tabular}{ll} 
		Im Studiengang: & CE \\		
	\end{tabular}	
	\\[3cm]
	\normalsize
	\begin{tabular}{ll}
	\textbf{Versuchsdatum:} & Montag, den 13.05.2024 \\
        \textbf{Erstellt von:} & Benjamin Brohs \\
        \textbf{Matrikel-Nr.:} & s0591345 \\
        \textbf{Gruppe:} & 7 \\
	\textbf{Dozent:} & Lars Bergenholtz
	\end{tabular}	
\end{titlepage}

\tableofcontents
\newpage

%%% Inhalt %%%%%%%%%%%%%%%%%%%%%%%%%%%%%%%%%%%%%%%%%%%%%%%%%%%%%%%%%%%%%%%%%%%

\section{Text} \label{sec:Text}
Text

%%% Verzeichnisse %%%%%%%%%%%%%%%%%%%%%%%%%%%%%%%%%%%%%%%%%%%%%%%%%%%%%%%%%%%%%%
\section{Anhang}
In diesem Abschnitt finden sich verschiedene Verzeichnisse der genutzten 
Ressourcen, sowie zusätzliche Anhänge von Ressourcen, die nicht direkt im 
Hauptteil des Dokuments enthalten sind.

\listoffigures
\listoftables
\listofcode

\iffalse
\section*{Arbeitsverteilung}

\subsection*{Benjamin Brohs}
\begin{mylist}
  \mylistentry{sec:intro}
\end{mylist}
\fi

%%% Anhänge %%%%%%%%%%%%%%%%%%%%%%%%%%%%%%%%%%%%%%%%%%%%%%%%%%%%%%%%%%%%%%%%%%%%

\iffalse
\begin{table}[h]
	\centering
	\begin{tabular}{|c|c|}
	\hline
	$U_{AB}$ & I \\
	\hline
	0   & 0 \\
	0,1 & ~ \\
	\hline
	\end{tabular}
	\caption{Messwerte nach 3.3.1}
	\label{tab:rx}
\end{table}
\fi

\iffalse
\begin{gfx}{Bodediagramm Phasengang Bandpass}{bodeband2}{}
    \begin{axis}
        [
        xlabel={$f$ [Hz] },
        ylabel={Phasenunterschied $\Delta \varphi_{12}$ [°] },
        grid=both,
        grid style={line width=.1pt, draw=gray!10},
        major grid style={line width=.2pt,draw=gray!50},
        xmin=20, xmax=100000,
        ymin=-90, ymax=90,
        width=11.89cm, height=6cm,
        every axis plot/.append style={very thick, mark=*},
        legend style={at={(0.5,-0.1)}, anchor=north, legend columns=-1},
        xtick={10, 100, 1000, 10000, 100000},
        ytick={-90,-75,-60,-45,-30,-15,0,15,30,45,60,75,90},
        xmode=log
        ]

        \addplot[
        scatter,
        only marks,
        point meta=explicit symbolic
        ] table[meta=label] {
            x y label
            100 74 1
            200 59 2
            341.49 44 10
            500 32 3
            1000 14 4
            2000 -1 5
            5000 -20 6
            10000 -39 7
            12847.92 -46 11
            20000 -59 8
            50000 -76 9
        };

        \addplot[
        no marks,
        thin,
        ] coordinates{
            (20, 86.51900995)
            (23.79683391, 85.86013154)
            (28.3144652, 85.07750427)
            (33.68973128, 84.1485388)
            (40.08544699, 83.04696273)
            (47.6953362, 81.74251804)
            (56.74989969, 80.20084926)
            (67.52339687, 78.38375782)
            (80.34215301, 76.25009007)
            (95.59444355, 73.7576336)
            (113.7422548, 70.86648447)
            (135.3352773, 67.54432971)
            (161.0275558, 63.7738094)
            (191.5972999, 59.56138864)
            (227.9704562, 54.94591252)
            (271.2487541, 50.00361115)
            (322.7430775, 44.84581114)
            (384.0131705, 39.60725845)
            (456.9148818, 34.42684488)
            (543.6563777, 29.42650423)
            (646.8650261, 24.69507184)
            (769.6669794, 20.2811201)
            (915.7818637, 16.1944633)
            (1089.635445, 12.41311546)
            (1296.493686, 8.892029277)
            (1542.622245, 5.571078989)
            (1835.476267, 2.381199607)
            (2183.926194, -0.751349392)
            (2598.526445, -3.902071813)
            (3091.835111, -7.145930748)
            (3678.794331, -10.554851)
            (4377.182883, -14.19424224)
            (5208.154703, -18.11787564)
            (6196.879622, -22.3608024)
            (7373.305756, -26.93078726)
            (8773.066621, -31.80001815)
            (10438.56046, -36.90028499)
            (12420.23448, -42.12536058)
            (14778.11285, -47.34270134)
            (17583.61485, -52.41271888)
            (20921.7181, -57.2098854)
            (24893.53254, -61.63886918)
            (29619.36296, -65.64164712)
            (35242.35304, -69.19586452)
            (41932.82109, -72.30765117)
            (49893.41894, -75.00259162)
            (59365.27018, -77.317439)
            (70635.27373, -79.29373889)
            (84044.79385, -80.97351325)
            (100000, -82.3966604)
        };

        \addplot [densely dotted, thin, no marks] coordinates {(341.485847695166,-90) (341.485847695166,90)};
        \node [below left] at (axis cs:341.485847695166,43.14655286) {$f_h$};
        
        \addplot [densely dotted, thin, no marks] coordinates {(2094.60836705993,-90) (2094.60836705993,90)};
        \node [below left] at (axis cs:2094.60836705993,0) {$f_0$};
        
        \addplot [densely dotted, thin, no marks] coordinates {(12847.9239797778,-90) (12847.9239797778,90)};
        \node [above right] at (axis cs:12847.9239797778,-43.14655286) {$f_t$};
    \end{axis}
\end{gfx}
\fi

%%%%%%%%%%%%%%%%%%%%%%%%%%%%%%%%%%%%%%%%%%%%%%%%%%%%%%%%%%%%%%%%%%%%%%%%%%%%%%%%

\end{document}