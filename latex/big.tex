\documentclass{article}
\usepackage[paperwidth=1000pt, paperheight=1000pt, margin=20pt]{geometry}
\usepackage{tikz}
\usepackage{tikz-uml}

\begin{document}

\begin{tikzpicture}
\umlclass[x=0,y=0,alias=UMLClass0,type={DatenObjekt}]{Vektor}
{
  x: Zahl \\
  y: Zahl \\
  z: Zahl \\
}
{
}
\umlclass[x=7,y=5,alias=UMLClass1]{World}
{
}
{
  block\_entfernen(position: Vektor): Bool \\
  block\_platzieren(position: Vektor): Bool \\
}
\umlclass[x=0,y=5,alias=UMLClass2]{Spielfigur}
{
  position: Vektor \\
  bewegung: Vektor \\
  am\_fliegen: Bool \\
}
{
  springen(): void \\
  sprinten(): void \\
  schleichen(): void \\
  fliegen(): void \\
  abbauen(): viod \\
  platzieren(): void \\
}
\umlclass[x=7,y=0,alias=UMLClass3]{Block}
{
  block\_id: Zahl \\
  textur\_id: Zahl \\
}
{
}
\umlclass[x=-7,y=5,alias=UMLClass4]{Inventar}
{
}
{
  item\_entfernen(item\_id: Zahl): Bool \\
  item\_hinzufügen(itme\_id: Zahl): Bool \\
}
\umlclass[x=-7,y=0,alias=UMLClass5]{ItemStack}
{
  item\_id: Zahl \\
  anzahl: Zahl \\
}
{
  ahzahl\_erhöhen(): void \\
  anzahl\_verringern(): void \\
}
\umlcompo[name=Relation0,geometry=--,mult1={1},pos1=0.2,mult2={*},pos2=0.8]{UMLClass1}{UMLClass3}
\umlassoc[name=Relation1,geometry=--,mult1={1},pos1=0.2,mult2={1},pos2=0.8]{UMLClass1}{UMLClass2}
\umlcompo[name=Relation2,geometry=--,mult1={1},pos1=0.2,mult2={*},pos2=0.8]{UMLClass2}{UMLClass4}
\umlaggreg[name=Relation3,geometry=--,mult1={1},pos1=0.2,mult2={*},pos2=0.8]{UMLClass4}{UMLClass5}
\end{tikzpicture}

\end{document}